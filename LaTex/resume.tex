\documentclass[]{article}

\usepackage{amsmath}
\usepackage{amssymb}
\usepackage{graphicx}
\usepackage{float}
\usepackage{multicol}
\usepackage{fancyhdr}
\fancyhead[L, CO] {}
\fancyhead[C, CO] {PartDiff - Résumé 2021}
\fancyhead[R, CO] {}
\usepackage[margin=1.5cm,landscape]{geometry}
\pagestyle{fancy}
\usepackage{mathtools} % for '\splitdfrac' macro
\usepackage[dvipsnames]{xcolor}
\usepackage{subfiles}
\usepackage{mdframed}


\begin{document}
\begin{multicols}{2}
\section{Table des matières}
\subsection{Première partie}
\begin{tabular}{ll}
Définition & 4\\
Types de capteurs & 5-13\\
Principes de mesure & 18-29\\
Méthode de mesure & 31-44\\
Circuits de référence & 47-52\\
Circuits amplificateurs & 53-60\\
Circuits de mesure & 62-64, 89-90\\
Bruit & 65-75, 81\\
Auto-zero & 76-78\\
Chopper amplifier & 79-80\\
Ponts & 85-88, 91\\
Démodulation & 92\\
Rétro-action & 93\\
Delta-sigma & 94-95\\
Conversions & 96-99\\
Calibration / configuration & 101-110\\
\end{tabular}
\subsection{Accéléromètres}
\begin{tabular}{ll}
Équations & 115-116, 144\\
Principe & 117-118\\
Marché & 120-125\\
Méthode de mesure & 126-132\\
Circuits & 133,135,136\\
\end{tabular}
\subsection{Gyroscopes}
\begin{tabular}{ll}
Équations & 148\\
Applications & 150-154\\
Méthode de mesure & 155-157, 160-162\\
Erreur de quadrature & 167\\
Modulation & 169-171\\
Random walk & 173-175\\
\end{tabular}
\subsection{Light sensor / ToF}
\begin{tabular}{ll}
Applications & 183, 197, 201\\
Types & 184, 189\\
Équations & 185-187\\
Sensibilité & 190-191\\
Ambient light sensor & 192-194\\
Mesure directe / indirecte & 200, 206-207\\
Types de diodes & 202-204\\
Multi-objets & 209-210\\
\end{tabular}
\subsection{Compas}
\begin{tabular}{ll}
Applications & 222-223\\
Équations & 219\\
Principe & 221, 226-231, 233-235, 237-239\\
Sensibilité & 224\\
Interférences & 241-247\\
\end{tabular}
\subsection{Energy harvesting}
\begin{tabular}{ll}
Définition / contexte & 252-255, 258\\
Sources d'énergie & 256\\
Contraintes & 257\\
Équations & 258,259\\
Mécanique & 260-264\\
Piezoélectrique & 265-270\\
Électro-magnétique & 271-275\\
Thermoélectrique & 276-282\\
Power management & 283-286\\
Stockage d'énergie & 287-291\\
\end{tabular}
\subsection{GNSS / GPS}
\begin{tabular}{ll}
Contexte & 294-301\\
Fonctionnement & 302-307\\
Types & 308
\end{tabular}





\subsection{Définition d'un smart sensor}
A smart sensor comprises the sensing head (measurement of one or 
more physical quantities) with associated electronic circuits and 
analog-to-digital conversion, high level signal processing and 
networking interfaces in a compact unit.\\
The term of smart sensing is also applied to multi-sensor networks with data processing generating complex measurement information


\end{multicols}
\end{document}